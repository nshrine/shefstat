\chapter{Discussion}\label{ch:discussion}
The were two primary questions addressed in this dissertation.
\begin{enumerate}
\item Given profiles of parasite counts per microlitre of blood with time from first dose of an antimalarial treatment, how do we derive an estimate of the time taken for 90\% of the parasites to be cleared from the blood?
\item Given our estimated times for parasite clearance, how do we determine if the average clearance time for subjects on one antimalarial treatment is significantly different to the average clearance time for subjects on an alternative treatment?
\end{enumerate}
The first question is not really one that can be formulated as a statistical hypothesis test, but is more a question of making a sensible and consistent estimate. The second question can be answered by the identification of a suitable test statistic.

The following discussion is primarily concerned with how these questions have been answered, but also goes on to look at some of the wider issues raised concerning analysis of data of this type.

\section{Derivation of clearance times}

\subsection{The nature of the problem}
In order to estimate the time for the parasite count to reach 90\% of its pre-dose level we need to know the nature of how the parasite count changes with time. Whether it falls monotonically, for example, follows a linear, or more complicated trend. The observed behaviours were summarised in section \ref{sec:behaviours} and can be seen in Figures \ref{raw1} and \ref{raw2} on pages \pageref{raw1} and \pageref{raw2}. One problem with this data is that we do not know how much of the erratic variation in parasite count per microlitre recorded genuinely reflects a sharp change in the parasite load of the subject and how much is due to error in the counting procedure. It is reported for patients taking antimalarial drugs that ``\textit{parasitemia may rise alarmingly in the hours following treatment}'' \cite{white}. However, it was reported by our contact at GSK that there may be a large variability in parasite counts simply due to the choice of the ``suitable'' area of the blood sample slide chosen by microscopists for parasite counting. Nevertheless, it seems that in the region around the PC90 level that the fall in parasite count is fairly smooth and not nearly as erratic as at earlier times. This can be seen in Figure \ref{comprawlog} on \pageref{comprawlog}. Hence, it seems that estimation between data recording times is more predictable in the PC90 region than at earlier times and that early erratic variation in the count is irrelevant. In this respect PC90 would seem a sensible choice of endpoint.
\subsection{Choice of interpolation method}
\subsubsection*{Logistic modelling}
We were told by GSK that logistic modelling had already been used for PC90 estimation for this data and suggested aims for this dissertation were the investigation of the logistic approach such as how to choose appropriate starting parameters for the fitting. This was done, however it was found that the logistic model simply did not have the appropriate shape for modelling the parasite-count profiles of some subjects. It is also not clear as to why a model for the whole count profile must be fitted in order to make an estimate within a narrow time window between observations that bracket the time of interest. It could be argued that the two observations either side of the PC90 time could be subject to a relatively large error and therefore the behaviour should be averaged over a wider timescale. However, this could be more suitably achieved with some sort of spline regression as it seems innappropriate for large fluctuations at early times to influence the estimate at later times when the count profile is relatively smooth.

Some problems with logistic regression are highlighted by Wootton {\it et al.}  who decided to exclude subjects whose parasite count profile could not be adequately fitted by a logistic model \cite{wootton}. This could, of course, bias the analysis if the departure from a logistic shape is correlated with experimental factors. In this study it was found that the 8 subjects out of the 43 for whom an adequate logistic fit could not be obtained were all from the single-drug group ($p<0.005$ under the null hypothesis of equal probability in both treatment groups of a poor fit). Wootton {\it el al}. however found that the subjects they excluded due to ``poor model fit'', of whom there were only 6, were fairly evenly distributed between the treatment groups. Nevertheless, a perhaps better approach than excluding the data would be to make a reasonable estimate - presumably any ``reasonable'' estimate would be bracketed by the time of the last count above 10\% of the pre-dose level and the first count below this - and then weight this datum by an estimate of its uncertainty. Of the literature studied, Wootton {\it et al}. is the only one that uses non-linear logistic regression for estimating PC90.

\subsubsection*{Cubic polynomial modelling}
If we decide that the PC90 estimate should be based on an overall trend in the parasite count rather than just a localized interpolation then it has been shown that linear cubic polynomial regression is more effective than non-linear logistic regression. The cubic polynomial model has enough flexibility to model either an approximately linear decline in parasite count or, if necessary, a highly curved trend with a point of inflection. It can be seen that the cubic fits obtained (Figure \ref{cubics}, page \pageref{cubics}) are at least as good as the logistic ones (Figure \ref{logistics}, \pageref{logistics}), apart from modelling the final decay to the zero level, but this isn't really relevant to the PC90 estimation. It could be argued that a cubic curve isn't appropriate or ``realistic'' as it doesn't reflect a transition between two levels of an initial high count to eventual low parasite counts. However, the aim here is not to model the whole process, but simply find a suitable interpolation that models the behaviour around the PC90 region. 

Another advantage is that linear regression can be performed analytically with no need to use numerical methods involving appropriate choice of starting parameters.

\subsubsection*{Log-linear interpolation}
The advantages of linear interpolation are that it is simple and gives the estimate most influenced by the data in the region of interest. It is the method chosen by Carmello {\it et al}. \cite{carmello} and Newton {et al}. \cite{newton}. It can be seen in Figures \ref{pc90-agree} to \ref{pc90-nofit} on pages \pageref{pc90-agree} to \pageref{pc90-nofit} that is produces estimates that closely match the cubic polynomial estimates.

\subsection{Assessing the accuracy of the PC90 estimates}
Assessing the accuracy of the PC90 estimates is not straightforward in that we do not have an idea of what the typical idealised parasite count profile should look like, if there is such a thing. As mentioned already, we do not know if fluctuations from a smooth trajectory are experimental counting errors or genuine sudden changes in the parasite count, which may occur due to 

RESIDUALS AND OUTLIERS 183 and 500

\section{Analysis of clearance times}

ANY STUDIES WITH DIFFERENT RESPONSE BETWEEN SEXES?
\section{Alternative analyses}
svensson poo poos PC90, PC50 as not representative of time course \cite{svensson}.