\chapter{Discussion and conclusions}\label{ch:discussion}
The were two primary questions addressed in this dissertation.
\begin{enumerate}
\item Given profiles of parasite counts per microlitre of blood with time from first dose of an antimalarial treatment, how do we derive an estimate of the time taken for 90\% of the parasites to be cleared from the blood?
\item Given our estimated times for parasite clearance, how do we determine if the average clearance time for subjects on one antimalarial treatment is significantly different to the average clearance time for subjects on an alternative treatment?
\end{enumerate}
The first question is not really one that can be formulated as a statistical hypothesis test, but is more a question of making a sensible and consistent estimate. The second question can be answered by the identification of a suitable test statistic.

The following discussion is primarily concerned with how these questions have been answered, but also goes on to look at some of the wider issues raised concerning analysis of data of this type.

\section{Derivation of clearance times}

\subsection{The nature of the problem}
In order to estimate the time for the parasite count to reach 90\% of its pre-dose level we need to know the nature of how the parasite count changes with time. Whether it falls monotonically, for example, follows a linear, or more complicated trend. The observed behaviours were summarised in section \ref{sec:behaviours} and can be seen in Figures \ref{raw1} and \ref{raw2} on pages \pageref{raw1} and \pageref{raw2}. One problem with this data is that we do not know how much of the erratic variation in parasite count per microlitre recorded genuinely reflects a sharp change in the parasite load of the subject and how much is due to error in the counting procedure. It is reported for patients taking antimalarial drugs that ``\textit{parasitemia may rise alarmingly in the hours following treatment}'' \cite{white}. However, it was reported by our contact at GSK that there may be a large variability in parasite counts simply due to the choice of the ``suitable'' area of the blood sample slide chosen by microscopists for parasite counting. Nevertheless, it seems that in the region around the PC90 level that the fall in parasite count is fairly smooth and not nearly as erratic as at earlier times. This can be seen in Figure \ref{comprawlog} on \pageref{comprawlog}. Hence, it seems that estimation between data recording times is more predictable in the PC90 region than at earlier times and that early erratic variation in the count is irrelevant. In this respect PC90 would seem a sensible choice of endpoint.
\subsection{Choice of interpolation method}
\subsubsection*{Logistic modelling}
We were told by GSK that logistic modelling had already been used for PC90 estimation for this data and suggested aims for this dissertation were the investigation of the logistic approach such as how to choose appropriate starting parameters for the fitting. This was done, however it was found that the logistic model simply did not have the appropriate shape for modelling the parasite-count profiles of some subjects. It is also not clear as to why a model for the whole count profile must be fitted in order to make an estimate within a narrow time window between observations that bracket the time of interest. It could be argued that the two observations either side of the PC90 time could be subject to a relatively large error and therefore the behaviour should be averaged over a wider timescale. However, this could be more suitably achieved with some sort of spline regression as it seems innappropriate for large fluctuations at early times to influence the estimate at later times when the count profile is relatively smooth.

Some problems with logistic regression are highlighted by Wootton {\it et al.}  who decided to exclude subjects whose parasite count profile could not be adequately fitted by a logistic model \cite{wootton}. This could, of course, bias the analysis if the departure from a logistic shape is correlated with experimental factors. In this study it was found that the 8 subjects out of the 43 for whom an adequate logistic fit could not be obtained were all from the single-drug group ($p<0.005$ under the null hypothesis of equal probability in both treatment groups of a poor fit). Wootton {\it el al}. however found that the subjects they excluded due to ``poor model fit'', of whom there were only 6, were fairly evenly distributed between the treatment groups. Nevertheless, a perhaps better approach than excluding the data would be to make a reasonable estimate - presumably any ``reasonable'' estimate would be bracketed by the time of the last count above 10\% of the pre-dose level and the first count below this - and then weight this datum by an estimate of its uncertainty. Of the literature studied, Wootton {\it et al}. is the only one that uses non-linear logistic regression for estimating PC90.

\subsubsection*{Cubic polynomial modelling}
If we decide that the PC90 estimate should be based on an overall trend in the parasite count rather than just a localized interpolation then it has been shown that linear cubic polynomial regression is more effective than non-linear logistic regression. The cubic polynomial model has enough flexibility to model either an approximately linear decline in parasite count or, if necessary, a highly curved trend with a point of inflection. It can be seen that the cubic fits obtained (Figure \ref{cubics}, page \pageref{cubics}) are at least as good as the logistic ones (Figure \ref{logistics}, \pageref{logistics}), apart from modelling the final decay to the zero level, but this isn't really relevant to the PC90 estimation. It could be argued that a cubic curve isn't appropriate or ``realistic'' as it doesn't reflect a transition between two levels of an initial high count to eventual low parasite counts. However, the aim here is not to model the whole process, but simply find a suitable interpolation that models the behaviour around the PC90 region. 

Another advantage is that linear regression can be performed analytically with no need to use numerical methods involving appropriate choice of starting parameters.

\subsubsection*{Log-linear interpolation}
The advantages of linear interpolation are that it is simple and gives the estimate most influenced by the data in the region of interest. It is the method chosen by Carmello {\it et al}. \cite{carmello} and Newton {et al}. \cite{newton}. It can be seen in Figures \ref{pc90-agree} to \ref{pc90-nofit} on pages \pageref{pc90-agree} to \pageref{pc90-nofit} that is produces estimates that closely match the cubic polynomial estimates.

\subsection{Assessing the accuracy of the PC90 estimates}
Assessing the accuracy of the PC90 estimates is not straightforward in that we do not have an idea of what the typical idealised parasite count profile should look like, if there is such a thing. As mentioned already, we do not know if fluctuations from a smooth trajectory are experimental counting errors or genuine sudden changes in the parasite count. Either way, the accuracy of our PC90 estimate will be related to how stable, and therefore predictable, the count trajectory is around PC90.

It was decided, by looking at the estimates graphically and making a judgement about where the parasite count is most likely to lie in-between the counts either side of PC90, that the log-linear interpolation method gives the most believable interpolated estimate. For example, subjects 288 and 509 in Figure \ref{pc90-bad} on page \pageref{pc90-bad}, where the cubic and logistic estimates seem to have deviated from the local trend, influenced by data far from the PC90 region. However, there is a case to be made that the cubic and logistic methods have made a better estimate, less influenced by localized variation, for subjects 490 and 500 in the same Figure.

There is no evidence, using a paired $t$ test to reject the hypothesis that the cubic regression and log-linear interpolation methods produce the same estimate on average. Furthermore, if we fit a mixed effects model with subject and PC90 estimation method as random effects (centre, sex and treatment being the fixed effects), then the standard deviation of the square root of PC90 between subjects is 5 times larger than between methods (page \pageref{methods.lme}). In this respect if we believe that our methods of estimating PC90 are unbiased then the accuracy is sufficient when taken in context with the larger ``error'' due to variation between subjects in the same fixed-effects group.

The logistic method gives estimates 32 and 49 minutes longer on average than the cubic and log-linear methods respectively. Examination of the fitted curves suggests that the logistic method is overpredicting the clearance times, but there does not appear to be any obvious bias in the cubic and log-linear methods.

It is unclear how to quantify the error of our estimation method at this stage. One could use the standard deviation between estimation methods, but then methods could be in close agreement but all be wrong in the same way. It would seem however that the error in the estimation method is smaller than the variance between subjects. 

One post-hoc method for assessing the accuracy of the estimation method is to look at the residuals from our modelling of the effect of experimental factors on PC90 and see if any outliers correspond to dubious PC90 estimates. However, our 3-way ANOVA models by centre, sex and treatment do not show any clear outliers with the largest magnitude residual being only $2.4\hat{\sigma}$. Consequently, we can derive little about the accuracy of PC90 estimation via this route.

\section{Analysis of clearance times}
\subsection{Choice of model}
It is clear that with our continuous response variable PC90 and categorical factors centre, sex and treatment, that a 3-way ANOVA model is the logical model to investigate first. ANOVA assumes independently normally distributed, homoscedastic 
errors. The residuals from the initial analysis were found to be correlated with fitted value and heteroscedastic between treatment groups. Accordingly a square root transformation of the PC90 dependent variable was used with least squares fitting, weighted by the inverse of the variance of the square root of PC90 of each treatment group. In this way the residuals of the ANOVA model now fulfilled, to a good approximation, the requirements of being normally distributed and homoscedastic.

\subsection{Results for the treatment effect}
No evidence was found to reject the hypothesis of the centre having no effect on PC90. Accordingly, centre was removed from the model and a 2-way ANOVA model by sex and treatment was fitted. It was found that there is some evidence to reject the hypothesis of there being no interaction of sex and treatment that effects PC90 ($p<0.05$). The primary results of the analysis are that there is no evidence to reject the hypothesis that PC90 is the same for male patients on either the single or combined treatment and that female subjects on the combined treatment have a mean PC90 13 hours shorter than those on the single treatment with a 95\% confidence interval of (5, 23) hours shorter.

Although there was a large variation in pre-dose parasite counts (median 23427; quartiles 15314, 31098; range 8500, 196029), there was no evidence to reject the hypothesis that PC90 is independent of pre-dose parasite count (ANCOVA model with pre-dose count as covariate).

The parametric ANOVA results were compared to non-parametric statistical tests using resampling methods, which gave the same results for hypothesis tests of centre, sex and treatment effects and a 95\% confidence interval for the decrease in PC90 for female subjects on the combined treatment of (6.5, 22.5) hours.

\subsection{Comparison with similar studies}
There were several studies similar to this one found in the the literature. The most directly comparable is the study of Wootton {\it et al}. \cite{wootton} who present the analysis of a clinical trial to test the effectiveness of a single drug chlorproguanil-dapsone (CPG-DDS) with its use in combination with 3 different doses of another drug, artesunate. They analyse PC90 as a primary endpoint and various secondary endpoints. They find a mean time to PC90 for subjects on CPG-DDS alone of 19.1 hours; we find a mean time to PC90 of 20 hours (95\% CI: 12, 30) for female subjects on the single treatment, but only 11 hours (95\% CI: 6, 18) for male subjects on the single treatment. They find a decrease in PC90 for subjects on the combined treatment of 6.6, 10.7 and 10.3 hours for 1, 2 and 4 mg/kg of artesunate taken in combination respectively. We find a reduction in PC90 of 13 hours (95\% CI: 5, 23) for female subjects on the combined treatment, but no evidence to reject the hypothesis that male subjects on both the single and combined treatments have the same mean PC90. The improvement of the combined treatment over the single-drug treatment we observed for female subjects is comparable to that Wootton observed in their experiment for all subjects on the higher doses of artesunate. Of course, this is a superficial comparison as we do not know anything about the protocol or drugs used for our data, but it generally supports the conclusion of Wootton {\it et al} that combined drug therapy reduces the clearance time, albeit only in female subjects in our study. The do not mention any effect of sex.

Other literature on antimalarial treatment did not report any difference in response between sexes \cite{carmello, newton, svensson, vries}.

\subsection{Interaction of sex and treatment}
In this study it should be noted the lack of apparent improvement for male subjects on the combined treatment is due to them having PC90 times on the single treatment that are similar to those for both male and female subjects on the combined drug treatment. The relative levels of the four sex-treatment groups can be seen in Figure \ref{pc90interaction} on page \pageref{pc90interaction}. It is fairly clear that the primary difference between the groups is that female subjects on the ``alone" single-drug treatment have higher mean clearance times than the other 3 groups, and that it is this that results in the conclusion that there is a treatment-sex interaction that affects clearance time. No similar observation has been found in the literature, which leads us to look more carefully at the validity of this conclusion. There is no obvious imbalance in the pre-dose counts between treatments, although there is a larger variance in pre-dose counts among subjects allocated to the single treatment group.

Looking at Figure \ref{allaov} on page \pageref{allaov} and Figure \ref{fdfitted} on page \pageref{fdfitted} it can be seen that the mean count for female subjects on the single treatment remains at a high level for longer than male subjects on the single treatment.

The sex-treatment interaction is confirmed by non-parametric ANOVA by ranks and also by resampling methods.

Therefore, we can be reasonably confident that we have good evidence to reject the hypothesis of no interaction between sex and treatment if we take the data we were given at face value. It could be that there is some bias in the data between sexes due to a protocol violation. If the observed interaction is not due to a real interaction between sex and treatment then we cannot determine alternative reasons for this observation from this data alone.

\section{Alternative analyses}
Svensson {\it et al}.  comment on analysis of antimalarial treatments by PC90 and PC50: ``
...\textit{these parameters do not 
describe the time course of the effect and might not be 
adequate in describing the efficacy of fast-acting anti-malarials like the artemisinin compounds unless para- 
sites are frequently counted} \cite{svensson}. They 

\section{Review and future recommendations}